\documentclass[letter,11pt,oneside]{article}

%%% (occur "\\(\\\\[a-z]*section\\|appendix\\|input\\|\\<include\\>\\)")

%%\documentclass[11pt,twocolumn]{article}
%%\usepackage[inline]{asymptote}   %% Inline asymptote diagrams
%%\usepackage{wglatex}             %% Use this one and kill others.
\usepackage{color}               %% colored letters {\color{red}{{text}}
\usepackage[listings,breakable]{tcolorbox}
\usepackage{fancyhdr}            %% headers/footers
%%\usepackage{fancyvrb}            %% headers/footers
\usepackage{datetime}            %% pick up tex date time 
\usepackage{lastpage}            %% support page of ...lastpage
\usepackage{times}               %% native times roman fonts
\usepackage{textcomp}            %% trademark
\usepackage{amssymb,amsmath}     %% greek alphabet
\usepackage{parskip}             %% blank lines between paragraphs, no indent
\usepackage{shortvrb}            %% short verb use for tables
\usepackage{lscape}              %% landscape for tables.
\usepackage{longtable}           %% permit tables to span pages wg-longtable
\usepackage{url}                 %% Make URLs uniform and links in PDFs
\usepackage{enumerate}           %% Allow letters/decorations for enumerations
\usepackage{endnotes}            %% Enhance footnotes/endnotes
\usepackage{listings}            %% Make URLs uniform and links in PDFs
\pdfadjustspacing=1                %% force LaTeX-like character spacing
%%\usepackage{geometry}            %% allow margins to be relaxed
%%\usepackage{wrapfig}             %% permit wrapping figures.
%%\usepackage{subfigure}              %% images side by side.
%%\geometry{margin=1in}            %% Allow narrower margins etc.
\usepackage[T1]{fontenc}         %% Better Verbatim Font.
\renewcommand*\ttdefault{txtt}   %% 
\usepackage[bookmarks]{hyperref} %% Make huperlinks within a PDF
%%\usepackage{natbib}   %% bibitems

%% include background image (wg-document-page-background) 

\usepackage{graphicx}            %% Include pictures into a document
%% (wg-texdoc-inserttikz)


\def\documentisdraft{NOTDRAFT}

%% (wg-texdoc-isdraft)
%% (wg-texdoc-insert-fancy-headers)

%%\usepackage[bookmarks]{hyperref} %% Make huperlinks within a PDF
%%\usepackage{makeidx}             %% Make an index uncomment following line
%%\makeindex                       %%.. yeah this one, too. index{key} in text
%%



\definecolor{verbcolor}{rgb}{0.6,0,0}
\definecolor{darkgreen}{rgb}{0,0.4,0}
\newcommand\debate[1]{\textcolor{darkgreen}{DEBATE: #1} \marginpar{\textcolor{red}{DEBATE} }}
\newcommand{\ltodo}[2]{\marginpar{\textcolor{red}{ACTION: #1}\endnote{#2}}}
\renewcommand{\thefigure}{\thesection-\arabic{figure}}
\newcommand{\menu}{\ensuremath{\;\rightarrow\;}}
\newcommand{\dhl}[1]{{\color{verbcolor}{\texttt#1}}}
\definecolor{wglightgreen}{rgb}{0.88, 0.58, 0.88}
\newcommand{\wgtextbox}[1]{\noindent\fcolorbox{darkgreen}{wglightgreen}{%
    \minipage[t]{\dimexpr0.80\linewidth-2\fboxsep-2\fboxrule\relax}
        {#1}
    \endminipage}}
%%(wg-add-inline-images)  %% add inline images to the mix
\tcbset{colback=yellow!10!white}  % tcolorbox background


%%Begin User Definitions: Hint: ~/.latex.defs and  latex.defs  
%%End User Definitions:
%%(wg-texdoc-adjust-paper-width)
%% (wg-texdoc-insert-hypersetup)


%%%%%%%%%%%%%%%%%%%%%%%%%%%%%%%%% PAGE SIZE %%%%%%%%%%%%%%%%%%%%%%%%%%%%%%%%%%%%%%%%%%%%
\pagestyle{fancy}
\usepackage[paperheight=7.125in,paperwidth=9.5in,footskip=.05in,margin=.75in,heightrounded]{geometry}
% (iv (setq tmp (/ (* 3.0 9.5) 4.0 )))   7.125
\fancyhf{}
\cfoot{{\tiny Page \thepage \hspace{1pt}}}

%%%%%%%%%%%%%%%%%%%%%%%%%%%%%%%%%%%%%%%%%%%%%%%%%%%%%%%%%%%%%%%%%%%%%%%%%%%%%


\begin{document}


%% (wg-latex-pretty-title-page)
%% (wg-texdoc-titleblock)


\title{ODroid KStars/Ekos/libindi Astronomy}
\author{Wayne Green}
\date{\today}
\maketitle

\begin{abstract}

  The ODroid OU4 is an 8-core Arm processor, here tricked out with a 64G
  eMMC card preloaded with Ubuntu 18.04. These instructions are how to
  load the ODroid to support KStars/Ekos/libindi for astronomy. It is more
  of a crib sheet. As an exercise the popular ds9 program compiled on the
  ODroid. 

\end{abstract}

%%%%%%%%%%%%%%%%%%%%%%%%%%%%%%%%%%%%%%%%%%%%%%%%%%%%%%%%%%%%%%%%%%%%%%%%%%%%%
%% table of contents
%%%%%%%%%%%%%%%%%%%%%%%%%%%%%%%%%%%%%%%%%%%%%%%%%%%%%%%%%%%%%%%%%%%%%%%%%%%%%
 
\pagenumbering{roman}   % i,ii,etc
%%\pagenumbering{gobble}   %ignore page numbers for a while
\pdfbookmark[0]{Table of Contents}{MyTOC} % if usepackage{hyperref} in use.
\tableofcontents
%%\listoffigures
%%\listoftables
\newpage


\setcounter{section}{1}
\pagenumbering{arabic}

\ifx\documentisdraft\drafttest
\linenumbers    %%%%%%%%%%%%% DRAFT
\fi

\section*{Overview}

First acquire the ODroid XU4, in my case the one with the fan. Get a
decent $\mu$SSD drive -- a Verbatim Premium 10 32G or equivalent.  I
used BalenEtcher on a Win10 to load the image.

Boot the ODroid, expand the filesystem to the whole card and bring the
image up to date.  Below are notes I took and rearranged to be in the
order of how you should do things. For example I loaded VIM (the vi
editor) early, locate\endnote{locate - Unix utility that builds a
  'database' of every file and makes searching a machine fast. When
  you add things, run \dhl{sudo updatedb} to manually rebuild the db.}
late (because it is easy to load.)  The next logical step is to get
the all the development installs done -- in this case I ran the INDI
part first.


I prefer to use ds9 as an image viewer and wanted it installed on the ODroid.
I did a \dhl{sudo apt-get install saods9} -- and the packaged program
crashs when loading a fits file iamge even though the program started up
fine. I downloaded the source, but the stock OS image was missing
quite a few libraries. One by one they were uncovered, and the
\dhl{apt-get install} lines placed to be right after the boot.
But in my actual order, I did Section \ref{sec:indipart} before
the ds9 libraries (I didn't know I needed them at that point).

I want the X11 functionalty, and I want Ekos to drop images where
I want them. I installed \dhl{NFS}, and took notes on the steps
to tie them to my main computer in Section \ref{sec:nfs}.


\subsection{Things to do once}

Using the \dhl{ssh} command, remote login to the odroid on your main
machine. You can do these things at the ODroid command prompt as well
if you like. If you are doing another odroid for the first time, then
you may have to remove its \dhl{ssh-key} -- the message will
tell you what to do:

\hskip 1cm \dhl{   ssh-keygen -f "/home/wayne/.ssh/known\_hosts" -R "odroid.local"}

Here is some code to cut/paste:

\begin{tcolorbox} %[breakable,colback=yellow!5!black]
\begingroup \fontsize{10pt}{10pt}
\selectfont
\begin{verbatim} 
# If this is a different odroid might have to remove the
# old hash from the .ssh/known_hosts file.
#ssh-keygen -f "/home/wayne/.ssh/known_hosts" -R "odroid.local"
ssh -l odroid odroid.local  # see remove old key above

# fix up the ~/.bashrc to git rid of colors. (hard to see)
echo "export PS1='(\j) \h \w [\!] '" >> ~/.bashrc
echo "alias ls='ls -FC'"            >> ~/.bashrc
echo "
\end{verbatim}
\endgroup
\end{tcolorbox}

\subsection{Initialize for KStars}

Here we remote login via ssh into the odroid, and follow
the commands below. It may require a reboot or two.
This is for a first time load. A short cut for the
huge KStars libraries and some default stuff you may
want will be developed below. (\today)

\begin{tcolorbox} %[breakable,colback=yellow!5!black]
\begingroup \fontsize{10pt}{10pt}
\selectfont
\begin{verbatim} 
#Setup
ssh -l odroid odroid.local  # see remove old key above

sudo apt-get update
sudo apt-get upgrade              # N - keep the provider's /etc/apt-fast.conf
sudo apt-get dist-upgrade
sudo apt-get install -y vim       # because,,, vi!
sudo apt install -y git           # because,,, git! (astrometry.net)

# Load these packages now, get it over with for ds9.
# Handy anyway.
sudo apt-get install -y locate       # bacause,,, locate!
sudo apt-get install -y libx11-dev   # needed for ds9
sudo apt-get install -y zlib1g-dev   # needed for ds9
sudo apt-get install -y libxml2-dev  # needed for ds9
sudo apt-get install -y libxslt1-dev # needed for ds9
sudo apt-get install -y autoconf     # needed for ds9

# for astrometry.net
sudo apt-get install -y swig
sudo apt-get install -y  python3-dev python3-pip python3-virtualenv
\end{verbatim}
\endgroup
\end{tcolorbox}


\subsection{KStars/EkosINDI part}  \label{sec:indipart}

Load the packages, and then using the KStars interface initialize the
site location etc, and load the internal things like lots of Tycho
stars. Add the Atik support libraries.

\begin{tcolorbox} %[breakable,colback=yellow!5!black]
\begingroup \fontsize{10pt}{10pt}
\selectfont
\begin{verbatim} 
# Indi/KStars
sudo apt-add-repository -y ppa:mutlaqja/ppa
sudo apt-get update
sudo apt-get install indi-full
sudo apt-get install kstars-bleeding

Atik Cameras:
sudo add-apt-repository -y ppa:mutlaqja/ppa
sudo apt-get update
sudo apt-get install indi-atik
sudo apt-get install libindi1     # moonlite NiteCrawler focuser

\end{verbatim}
\endgroup
\end{tcolorbox}

\subsection{Remote X Desktop}  \label{sec:xrdp}

This part will be idle most of the time -- but still cycle-sucking.
I did it anyway.


\begin{tcolorbox} %[breakable,colback=yellow!5!black]
\begingroup \fontsize{10pt}{10pt}
\selectfont
\begin{verbatim} 
# https://linuxize.com/post/how-to-install-xrdp-on-ubuntu-18-04/
apt install xfce4 xfce4-goodies xorg dbus-x11 x11-xserver-utils
apt install xrdp 

systemctl status xrdp
adduser xrdp ssl-cert  
systemctl restart xrdp
ufw allow from 192.168.1.0/24 to any port 3389
ufw allow 3389
\end{verbatim}
\endgroup
\end{tcolorbox}

\subsection{SAOImage/ds9}  \label{sec:ds9}

Odroid:

\begin{tcolorbox} %[breakable,colback=yellow!5!black]
\begingroup \fontsize{10pt}{10pt}
\selectfont
\begin{verbatim} 
cd /Downloads
wget http://ds9.si.edu/download/source/ds9.8.1.tar.gz
tar xzvf ds9.8.1.tar.gz  # -> makes SAO
cd SAO
make unix/install > log0 2>&1
make >> log0 2>&1
tail -f log0  # ties the window up watching the make
\end{verbatim}
\endgroup
\end{tcolorbox}

This went on for quite a while, each failure ment tracking
down the missing library, adding that to the top of
this file, and going again. It was painful, fortunately
I knew ds9 when it was sane, and could decipher the
twisted package names.

Eventually:

\begin{tcolorbox} %[breakable,colback=yellow!5!black]
\begingroup \fontsize{10pt}{10pt}
\selectfont
\begin{verbatim} 
ssh -X odroid@odroid # from titan
ds9 theOdroidFits.fits # pops up an X display on Titan!
\end{verbatim}
\endgroup
\end{tcolorbox}
But, downloads bog down the ODroid, so we want INDI to
drop the file via the network onto \dhl{titan}! So
on to NFS.


\subsection{ODroid in the Network}  \label{sec:networking}

In this example, my main machine \dhl{titan} lives on
my network at \dhl{192.168.0.216}. Currently the ODroid
uses DHCP and is at \dhl{192.168.0.218} -- I maintain
long DHCP leases with renewal on my network host.

The ODroid is gig-Ether.

I added \dhl{192.168.0.216} to the odroid \dhl{/etc/hosts} file.
I added \dhl{192.168.0.218}  to titans \dhl{/etc/hosts} file.

Titan has the nfs-server installed, has been for years.

Titan has one-and-only one file that matters: the \dhl{/etc/exports}
file. It needs one line added to allow the ODroid to come in
as a user.

The ODroid needs to have the \dhl{nfs-common} package installed,
one line added to its /etc/fstab, and a little helper script
added to \dhl{/etc/network/if-up.d/fstab} to mount the nfs
connection during odroid-boot. (I suspect late rc2, before rc3).
I don't care, I want it ready with run-level 5 hits.


\subsection{NFS}  \label{sec:nfs}

Third things first: This machine is behind at least 3 routers
in my system, the last is the WAN gateway.

Getting NFS was a bit painful to implement and I need to dig into
some security issues. I should do the export on titan
ONLY for ODroid's IP address for example. It's been 30 years
since I had to do this for a living.

Firewalls need to be explored and opened. My IPtables are a bit goofy
on titan as it does Strange Things\texttrademark\; with Docker
containers, and the rest of my network.

When the camera takes an image, it will put the image into
\dhl{ordoid:/mnt/titantoday} that winds up on
\dhl{titan:/export/wayne/titantoday}. So, AS the camera
is downloading -- the network is churning and the file's write
finishes on \dhl{titan}. While the next hammering download
is taking place, titan does not feel the impact so much.

Note: This will have to change for high cadence work,
but it is not too bad.

NFS is a Unix/Unix network file sharing program. I trust it more than
SMB -- but SMB has made progress. It is the one I choose, as my
analysis routines are Unix based on a remote server anyway.

I suspect a NFS/Tunneling might be slower.

Or, I may implement a ``pull'' strategy from titan, to
get the files over when I want them. This means titan will
not see the camera download at all.

\begin{tcolorbox}[breakable]
\begingroup \fontsize{10pt}{10pt}
\selectfont
%%\begin{Verbatim} [commandchars=\\\{\}]
\begin{verbatim} 
# Now to get NFS working
#On odroid
# safer? sudo ufw allow from 192.168.0.216/32 to any port nfs
sudo ufw allow from 192.168.0.0/24 to any port nfs
sudo ufw status
sudo apt install nfs-common
---------------------------------
# odroid /etc/fstab
UUID=e139ce78-9841-40fe-8823-96a304a09859 / ext4 errors=remount-ro,noatime 0 1
LABEL=boot /media/boot vfat defaults 0 1
titan:/export/wayne/ /mnt/titantoday nfs rw,async,user,auto 0 0 

mount -a
# manual way... mount titan:/export/wayne/ /mnt/titantoday/

# add a config script to mount fs
cat >/etc/network/if-up.d/fstab
#!/bin/sh
mount -a
^D
chmod +x /etc/network/if-up.d/fstab
---

#On titan: edit /etc/exports
/export       192.168.0.1/24(rw,fsid=0,insecure,no_subtree_check,async)
/export/users 192.168.0.0/24(rw,nohide,insecure,no_subtree_check,async)
/export/wayne 192.168.0.0/24(rw,async,no_subtree_check,anonuid=1000,anongid=1000)
sudo exportfs -av

# I suspect the line CIDR /32 = netmask of 255.255.255.255 or precise macine.
# anonuid=1000,anongid=1000 is a bit concerning! It requires agreement in
# user's ids between machines and 1000 is default linux install.
/export/wayne 192.168.0.216/32(rw,async,no_subtree_check,anonuid=1000,anongid=1000)

\end{verbatim}
\endgroup
%% \end{Verbatim}
\end{tcolorbox}
\section{Summary/Things to do}

Three-D print a case that will hold on to something. Probably
my Alpy-600 to start. That fan is a bit questionable, but another
is hammering away at the back of the camera.

I still need to write the task that drives ds9 when new images
shows up. Did a lot of documentation on that as I was trying
to get that to work from inside a Docker container. Futile
I suspect, but I am sure learning networking on Docker's virtual
net.


\section{Operations}

It is possible to store images both 'locally' on the ODroid,
and remotely (see NFS above.)

the \dhl{\~/Desktop/Today} is where all the software
has been configured (in the past!) to use. At the
end of the night, Rename this and copy to local
machine as needed.

With NFS, the copies are done one-by-one in realtime.


\begin{tcolorbox} %[breakable,colback=yellow!5!black]
\begingroup \fontsize{10pt}{10pt}
\selectfont
%%\begin{Verbatim} [commandchars=\\\{\}]
\begin{verbatim} 
ssh -l odroid@odroid.local
...pw...

cd Desktop

ls -l Today          # see if exists,
mv Today when.Today  # move it to the 'right' date
mkdir Today          # make todays.
\end{verbatim}
\endgroup
%% \end{Verbatim}
\end{tcolorbox}

\section{Use with The Sky/X (TSX)} \label{sec:skyx}

The remote use requires the TCP/IP address of the
mount controler. TSK \dhl{Tools->TCP Server}, the port is
\dhl{3040} by default. On the same computer, set the
TCP/IP address to localhost \dhl{127.0.0.1}.

\url{https://www.indilib.org/devices/mounts/bisque-paramount.html}

\index{TSX!port} \index{TSK!TCP/IP}



\section{Astrometry.net Local}  \label{sec:astrommetry}

The main documentation is at: \index{AstrometryNet!Documentation}

\url{http://astrometry.net/doc/index.html}

The \dhl{Astrometry.net} package can be obtained via git.
\index{AstrometryNet!Build Instructions}

\begin{tcolorbox} %[breakable,colback=yellow!5!black]
\begingroup \fontsize{10pt}{10pt}
\selectfont
%%\begin{Verbatim} [commandchars=\\\{\}]
\begin{verbatim} 
mkdir -p ~/git/{clones,external}
mkdir -p ~/clones
cd ~/clones
wget http://heasarc.gsfc.nasa.gov/FTP/software/fitsio/c/cfitsio-3.48.tar.gz
tar xzvf cfitsio-3.48.tar.gz
cd cfitsio-3.48
./configure
make > log1 2>&1 &
tail -f log1
view Makefile # make sure install dir is right. Ha! prefix=/usr needed
sudo make prefix=/usr install

#git clone ....the clone from github
cd ~/git/external
git clone https://github.com/dstndstn/astrometry.net.git
cd astrometry.net
# in the background, run make and watch for errors with tail -f
make > log1 2>&1 &
tail -f log1
# ... take a minute! ...
# making this documentation, cfitsio was missing, see above
make > log2 2>&1 &
tail -f log2
# OK had to pause and install swig (above)
make > log3 2>&1 &
tail -f log3

# OK had to pause and install python dev (above)
make > log4 2>&1 &
tail -f log4

sudo make install INSTALL_DIR=/usr/local >install.log 2>&1
less install.log

# get some index files
# first-off download them to a local place on your
# network, then copy them here to the ODroid.
# get wayne's handy scripts.
\end{verbatim}
\endgroup
%% \end{Verbatim}
\end{tcolorbox}

\subsection{Index Files}

The pagkage is installed at a given root location, here
\dhl{/usr/local/share/astrometry}. The
data files are controled by a file called \dhl{etc/astrometry.cfg}.

\begin{tcolorbox} %[breakable,colback=yellow!5!black]
\begingroup \fontsize{10pt}{10pt}
\selectfont
%%\begin{Verbatim} [commandchars=\\\{\}]
\begin{verbatim} 
add_path /usr/local/astrometry/data
autoindex
\end{verbatim}
\endgroup
%% \end{Verbatim}
\end{tcolorbox}

The two lines use all the files in the \dhl{/usr/local/astrometry/data}
location. There are three strategies:

- Keep all the files together in the \dhl{data} directory.



Acquire the INDEX-42 files.

\begin{table}[h!]
%\phantomsection
%\addcontentsline{toc}{section}{ TOC CAPTION}
% \setlength{\belowcaptionskip}{6pt} % adjust space under caption abovecaptionskip
% \renewcommand{\arraystretch}{1.3} % adjust line spacing
%\small{
%\begin{minipage}{\textwidth}     % for footnotes in table.
%\caption[TOC]{Index File Scales}
\centering
\begin{tabular}{| l | l | r  r | r  r |}
%\MakeShortVerb{\|}
%\multicolumn{n}{fmt}{text for merged cols}
\hline
Scale &  File                 & \multicolumn{2}{| l |}{  Range}    &  \multicolumn{2}{| l |}{ Range }  \\
      &                       & \multicolumn{2}{| l |}{  (arcmin)} & \multicolumn{2}{| l |}{ (degrees) } \\
\hline
00    &  index-4200-*.fits    &     2.0  &    2.8 &           &    \\    
01    &  index-4201-*.fits    &     2.8  &    4.0 &           &    \\ 
02    &  index-4202-*.fits    &     4.0  &    5.6 &           &    \\ 
03    &  index-4203-*.fits    &     5.6  &    8.0 &           &    \\ 
\hline
04    &  index-4204-*.fits    &     8    &    11  &           &    \\ 
05    &  index-4205-*.fits    &     11   &    16  &           &    \\ 
06    &  index-4206-*.fits    &     16   &    22  &           &    \\ 
07    &  index-4207-*.fits    &     22   &    30  &           &    \\ 
08    &  index-4208.fits      &     30   &    42  &           &    \\ 
09    &  index-4209.fits      &     42   &    60  &           &    \\ 
10    &  index-4210.fits      &     60   &    85  &           &    \\ 
11    &  index-4211.fits      &     85   &   120  &      1.42 &   2.00    \\ 
\hline
12    &  index-4212.fits      &     120  &   170  &      2.00 &   2.83    \\ 
13    &  index-4213.fits      &     170  &   240  &      2.83 &   4.00    \\ 
14    &  index-4214.fits      &     240  &   340  &      4.00 &   5.67    \\ 
15    &  index-4215.fits      &     340  &   480  &      5.67 &   8.00    \\ 
16    &  index-4216.fits      &     480  &   680  &      8.00 &  11.33    \\ 
17    &  index-4217.fits      &     680  &  1000  &     11.33 &  16.67    \\ 
\hline
18    &  index-4218.fits      &     1000 &  1400  &     16.67 &  23.33    \\ 
19    &  index-4219.fits      &     1400 &  2000  &     23.33 &  33.33    \\ 
%% ones-based: \cline{a-b}
\hline
%%\DeleteShortVerb{|}
\end{tabular}
%%\end{minipage}    %% for footnotes  r@{.}l 
\caption[TOC]{Index File Scales}
\label{table:IndexFileScales}
%%} % end small etc
\end{table}


Make GAIA index files.
\url{https://github.com/LCOGT/gaia-astrometry-net}

Get Pre-made GAIA files for scales 00 through 07. 194 files.
\url{http://data.astrometry.net/5000/}



\section{Summary} \label{sec:summary}

The KStars planetarium packages leaves quite a bit to be desired,
but allows one to add their own libraries, and control cameras etc.
The ODroid/KStars/Ekos/INDI suite, with configuration tricks here
allows the odroid to happily live on the telescope without
keyboard/mouse/monitor. IT requires power and Ethernet from the
outside -- and connections to the instrument: USB for the camera
and perhaps USB for the filter wheel.




%%\appendix
%%\renewcommand \thesection{\Alph{section}}

%% use a bibitem approach to the references publications etc.
%% (wg-bibitem)

%%\clearpage
%%\addcontentsline{toc}{section}{References}
%%\renewcommand*{\refname}{My Bibliography and References}
%%\bibliographystyle{plain}	% bibliographystyle{apalike} and \usepackage{natbib}
%%\bibliography{MasterBib}	% expects file "MasterBib.bib"


%%\clearpage
%%\addcontentsline{toc}{section}{Index}
%%\printindex %% www.cs.usask.ca/resources/tutorials/latex/notes/toc-index.pdf

%%\begin{thebibliography}{80}
%%\usepackage{natbib}   %% bibitems
%%\end{thebibliography}

% /home/wayne/git/pre/pre.Odroid/doc/Odroid.tex

%% (wg-texdoc-endnotes)
\end{document}


#on odroid
# add titan to /etc/hosts
echo "192.168.0.216 titan" >> /etc/hosts

# add export info on titan/restart nfs
/export/wayne 192.168.0.0/24(rw,async,no_wdelay,no_root_squash,insecure_locks,sec=sys,anonuid=1025,anongid=100)
systemctl restart nfs-kernel-server


#Add to tail of /etc/fstab
titan:/export/wayne/Today /home/odroid/desktop/Today nfs rw,user,auto 0 0

---
https://linuxconfig.org/how-to-configure-a-nfs-file-server-on-ubuntu-18-04-bionic-beaver

#titan /etc/exports
/export       192.168.0.1/24(rw,fsid=0,insecure,no_subtree_check,async)
/export/users 192.168.0.0/24(rw,nohide,insecure,no_subtree_check,async)
#/export/wayne 192.168.0.0/24(rw,nohide,insecure,no_subtree_check,async)
/export/wayne 192.168.0.0/24(rw,async,no_wdelay,no_root_squash,insecure_locks,sec=sys,anonuid=1025,anongid=100)


Kevin AVX mount.
