\documentclass[letter,11pt,oneside]{article}
%%% HEREHEREHERE
%%% APPENDIX
%%% (insert (format "\n%s\n" (buffer-file-name)))
%%% (occur "\\(\\\\[a-z]*section\\|appendix\\|input\\|\\<include\\>\\)")

%%\documentclass[11pt,twocolumn]{article}
%%\usepackage[inline]{asymptote}       %% Inline asymptote diagrams
%%\usepackage{wglatex}                 %% Use this one and kill others.
\usepackage{color}                     %% colored letters {\color{red}{{text}}
\usepackage[listings,breakable]{tcolorbox}
\usepackage{fancyhdr}                  %% headers/footers
\usepackage{fancyvrb}                %% headers/footers
\usepackage{datetime}                  %% pick up tex date time
\usepackage{lastpage}                  %% support page of ...lastpage
\usepackage{times}                     %% native times roman fonts
\usepackage{textcomp}                  %% trademark
\usepackage{amssymb,amsmath}           %% greek alphabet
\usepackage{parskip}                   %% blank lines between paragraphs, no indent
\usepackage{shortvrb}                  %% short verb use for tables
\usepackage{lscape}                    %% landscape for tables.
\usepackage{longtable}                 %% permit tables to span pages wg-longtable
\usepackage{multicol}                  %% Enhance footnotes/endnotes
\usepackage{url}                       %% Make URLs uniform and links in PDFs
\usepackage{enumerate}                 %% Allow letters/decorations for enumerations
\usepackage{endnotes}                  %% Enhance footnotes/endnotes
\usepackage{listings}                  %% Make URLs uniform and links in PDFs
\pdfadjustspacing=1                    %% force LaTeX-like character spacing
%%\usepackage{geometry}                  %% allow margins to be relaxed
%%\usepackage{wrapfig}                 %% permit wrapping figures.
%%\usepackage{subfigure}               %% images side by side.
%%\geometry{margin=1in}                  %% Allow narrower margins etc.
\usepackage[T1]{fontenc}               %% Better Verbatim Font.
\renewcommand*\ttdefault{txtt}        %%
\usepackage[bookmarks,colorlinks=true]{hyperref} %% Make huperlinks within a PDF
\usepackage{natbib}                    %% bibitems

%% include background image (wg-document-page-background)

\usepackage{graphicx}            %% Include pictures into a document
%% (wg-texdoc-inserttikz)


\def\documentisdraft{NOTDRAFT}

%% (wg-texdoc-isdraft)
%% (wg-texdoc-insert-fancy-headers)

%%\usepackage[bookmarks]{hyperref} %% Make huperlinks within a PDF
%%\usepackage{makeidx}             %% Make an index uncomment following line
%%\makeindex                       %%.. yeah this one, too. index{key} in text
%%



\definecolor{verbcolor}{rgb}{0.6,0,0}
\definecolor{darkgreen}{rgb}{0,0.4,0}
\newcommand\debate[1]{\textcolor{darkgreen}{DEBATE: #1} \marginpar{\textcolor{red}{DEBATE} }}
\newcommand{\ltodo}[2]{\marginpar{\textcolor{red}{ACTION: #1}\endnote{#2}}}
\renewcommand{\thefigure}{\thesection-\arabic{figure}}
\newcommand{\menu}{\ensuremath{\;\rightarrow\;}}
\newcommand{\dhl}[1]{{\color{verbcolor}{\texttt#1}}}
\definecolor{wglightgreen}{rgb}{0.88, 0.58, 0.88}
\newcommand{\wgtextbox}[1]{\noindent\fcolorbox{darkgreen}{wglightgreen}{%
    \minipage[t]{\dimexpr0.80\linewidth-2\fboxsep-2\fboxrule\relax}
        {#1}
    \endminipage}}

\newcommand\snippet[2]{%
\begin{figure}[h!]
\begin{tcolorbox}[colback=yellow!15!white]
\begingroup \fontsize{10pt}{10pt}
\selectfont
\VerbatimInput{snippets/#1}
\endgroup
\end{tcolorbox}
\caption{#2}
\label{fig:KStarsPreliminary}
\end{figure}
}

\hypersetup{
colorlinks=true,
linkcolor=red,
citecolor=red,
urlcolor=blue,
pdfauthor = {Copyright(c) 2020. All rights reserved. Wayne Green},
pdftitle = {Astro Docker: Sextractor and Astrometry.net},
pdfsubject = {Photometry Reduction},
pdfkeywords = {photometry, astrometry},
pdfcreator = {LaTeX with hyperref package},
pdfproducer = {dvips + ps2pdf}}


%%(wg-add-inline-images)  %% add inline images to the mix


%%%%%%%%%%%%%%%%%%%%%%%%%%%%%%%%% PAGE SIZE %%%%%%%%%%%%%%%%%%%%%%%%%%%%%%%%%%%%%%%%%%%%
\pagestyle{fancy}
\usepackage[paperheight=7.125in,paperwidth=9.5in,footskip=.05in,margin=.75in,heightrounded]{geometry}
% (iv (setq tmp (/ (* 3.0 9.5) 4.0 )))   7.125
\fancyhf{}



%%%%%%%%%%%%%%%%%%%%%%%%%%%%%%%%%%%%%%%%%%%%%%%%%%%%%%%%%%%%%%%%%%%%%%%%%%%%%


\begin{document}


%% (wg-latex-pretty-title-page)

%%%%%%%%%%%%%%%%%%%%%%%%%%%%%%%%%%%%%%%%%%%%%%%%%%%%%%%%%%%%%%%%%%%%%%%%%%%%%
%%%%%%%%%%%%%%%%%%%%%%%%%%%% START PRETTY TITLE PAGE %%%%%%%%%%%%%%%%%%%%%%%%%%%%%%%%%
\pagecolor{blue!50}

\includegraphics[width=.9\textwidth]{images/SASGitTitle.png} \\
\vskip 1mm {\tiny git consortium}


\pagenumbering{gobble}   %ignore page numbers for a while

\newpage
\pagecolor{white}

%% (wg-texdoc-titleblock)

\setcounter{section}{0}
\pagenumbering{arabic}

\ifx\documentisdraft\drafttest
\linenumbers    %%%%%%%%%%%%% DRAFT
\fi

\newpage

\section*{Overview}

We use ``git'' to keep versions of images in a central ``cloud'' location.

Developers new to the project ``clone'' this image in its entirty.

It is possible to get just the recent images related to the project.

You make your local copy and do some work.
\vspace{-.15cm}
\begin{enumerate}\addtolength{\itemsep}{-0.5\baselineskip}
   \item   New files
   \item   Modify existing files
\end{enumerate}

Several modifications, until you are ready to ``add'' those changes
back into the mainstream. 

Then you ``commit'' to the files you added, adding a brief message
about what was done. This updates your local copy.

Then you ``push'' your changes to the remote location to be available
for all.

So:

\vspace{-.15cm}
\begin{enumerate}\addtolength{\itemsep}{-0.5\baselineskip}
   \item   Clone - make a copy of a repository, that is a repository in its own
             right.
   \item   Work - Do work in the directory with an eye to permenant changes.
   \item   Add  - Add the new/updated to the local list of things to make
            permenant in the local repository.
   \item   Commit - Commit those changes locally.
   \item   Push - Push the changes out to the rest of the world. Its on you
           to make sure you don't break anything.
   \item   Pull - Get other peoples changes down to the local level.
\end{enumerate}

With git, the main development ``thread'' is called the ``master
branch''\footnote{The current repository is called ``master'', the new move
is to call the main branch, well ``main''.} and is usually the stable 
path for the code. When new ideas are agreed upon, you create a ``branch'',
do the work/add/commit (locally) -- until you are happy. You may push
those changes up to the remote repository's ``branch'' of the same name --
issue a pull requiest; have your peers work on the code with changes etc.
At some point, the results of the branch are ``merged'' back into the ``main
branch'' and a ``release'' is made. This ``release'' is ``tagged'' with
a revision number, and life moves on.

Branches may be branched even deeper. This may allow two competing ideas
to duke it out -- surviver is merged upward.

Rinse and Repeat.

\subsection{Bits of Arcana}




\subsection{Issues}

We maintain ``private'' repositories:
\vspace{-.15cm}
\begin{enumerate}\addtolength{\itemsep}{-0.5\baselineskip}
   \item   Respect the priviledge of everyone's privacy
   \item   Yours is respected to the extent possible (hackers are good)
   \item   We will all agree to make the repository public, or not.
\end{enumerate}


Here we try to distill the capabilities of a program to meet a complex
issue of managing changes for projects in a way as to be 100\% safe
in retaining code, documentation, and data.

\newpage

\section{Get Started}

It is critical to have an agreed-upon way to put information into
a version control system. This respects other people's ability to
find things.

To join the SAS Spectrography GitHUB project:

\vspace{-.15cm}
\begin{enumerate}\addtolength{\itemsep}{-0.5\baselineskip}
   \item   Go to \url{https://github.com/}
   \item   Sign-up (please fill in the damned profile and provide a recognizable icon).
   \item   Send login name to project leader
\end{enumerate}

The project leader will add you to the list of people that use the
reposirory.

\section{GIT Version Control}

The ``git'' package was written in over several weeks, yes weeks,
because Linux Torvalds was disappointed with existing ``version
control'' systems. It went through a brief period of improvement
by the team that manages the roughtly 1e6 lines of the Linux Kernel.
It was released, and in minor ways, improved over the years.

Git forms the foundation for almost all serious code develoment.
It is not limited to code. 

The basic idea, is each file is copied to a ``main'' repository ``somewhere''.
The files are saved with a name that ``is'' the SHA1 hash code for
their contents -- offering a one in 1e48-ish chance of two different
files having the same name. A little side-table tracks the file's real
name. 

In a repository, such as the one on GitHub, has copies of each revision
of the file -- going back. 

You can make a complete copy of that repository onto your machine. That
repository is just as capable of being the main repo -- main by virtue
of being agreed-upon by social convention. 

A truly independent mirroring of important details that makes it
almost impossible\footnote{Barring stupidity that is in abundant
  supply.} for a ``bad actor'' to create Chaos\texttrademark.


\appendix
\renewcommand \thesection{\Alph{section}}

\section{Examples}

\subsection{Use the Browser to Add New Files}

Go to the GitHub repository and log in.

To create a new directory, the fast way, is to create a new file
MyNewDirName/README.md -- to start a README file (good idea anyway)
and put in some descriptive information about this thing.

Then simply browse down into the new directory, drag and drop
the new files ``up there''.



\section{Under the Hood}

%% use a bibitem approach to the references publications etc.
%% (wg-bibitem)

%%\clearpage
\addcontentsline{toc}{section}{References}
\renewcommand*{\refname}{My Bibliography and References}
\bibliographystyle{apalike}	% bibliographystyle{apalike} and \usepackage{natbib}
\bibliography{MasterBib}	% expects file "MasterBib.bib"



%%\begin{thebibliography}{80}
%%\usepackage{natbib}   %% bibitems
%%\end{thebibliography}

%%\clearpage
%%\addcontentsline{toc}{section}{Index}
%%\printindex %% www.cs.usask.ca/resources/tutorials/latex/notes/toc-index.pdf

% /home/wayne/iraf/smtsci/tex/SextractorDocker.tex

%% (wg-texdoc-endnotes)

%%%%%%%%%%%%%%%%%%%%%%%%%%%%%%%%%%%%%%%%%%%%%%%%%%%%%%%%%%%%%%%%%%%%%%%%%%%%%
% Support for endnotes
\begingroup
\renewcommand{\notesname}{\textcolor{red} {Action Items:}}
\parindent 0pt
\parskip 2ex
\phantomsection
\addcontentsline{toc}{section}{	extcolor{red} {Action Items:}}
\def\enotesize{\normalsize}
\theendnotes
\endgroup

\end{document}
